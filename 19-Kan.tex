\documentclass[DaoFP]{subfiles}
\begin{document}
\setcounter{chapter}{18}

\chapter{Kan Extensions}

If category theory keeps raising levels of abstraction it's because it's all about discovering patterns. Once patterns are discovered, it's time to study patterns formed by these patterns, and so on. 

We've seen the same recurring concepts described more and more tersely at higher and higher levels of abstraction. 

For instance, we first defined the product using a universal construction. Then we saw that the spans in the definition of the product were natural transformations. That led to the interpretation of the product as a limit. Then we saw that we can define it using adjunctions. We were able to combine it with the coproduct in one terse formula:
\[ (+) \dashv \Delta \dashv (\times) \]
Lao Tzu said: ``If you want to shrink something, you must first allow it to expand."

Kan extensions raise the level of abstraction even higher. Mac Lane said: ``All concepts are Kan extensions.''

\section{Closed Monoidal Categories}

\begin{itemize}
\item (co-) tensoring 
\item Kan extensions as the adjoints to functor pre-composition
\item (co-) end representation

\item (co-) limits as Kan extensions

\item adjunctions as Kan extensions

\item pointwise tensor product

\item Day convolution as Kan extension
\end{itemize}
\end{document}
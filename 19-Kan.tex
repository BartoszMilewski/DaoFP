\documentclass[DaoFP]{subfiles}
\begin{document}
\setcounter{chapter}{18}

\chapter{Kan Extensions}

If category theory keeps raising levels of abstraction it's because it's all about discovering patterns. Once patterns are discovered, it's time to study patterns formed by these patterns, and so on. 

We've seen the same recurring concepts described more and more tersely at higher and higher levels of abstraction. 

For instance, we first defined the product using a universal construction. Then we saw that the spans in the definition of the product were natural transformations. That led to the interpretation of the product as a limit. Then we saw that we can define it using adjunctions. We were able to combine it with the coproduct in one terse formula:
\[ (+) \dashv \Delta \dashv (\times) \]
Lao Tzu said: ``If you want to shrink something, you must first allow it to expand."

Kan extensions raise the level of abstraction even higher. Mac Lane said: ``All concepts are Kan extensions.''

\section{Closed Monoidal Categories}

We've seen how a function object can be defined as the right adjoint to the categorical product:
\[ \cat C (a \times b, c) \cong \cat C (a, [b, c]) \]
Here I used the alternative notation $[b, c]$ for the exponential $c^b$. 

An adjunction between two functors can be thought of as them being half-inverse of each other. Their composition is related to the identity functor through unit and counit. For instance, if you squint hard enough, the counit of the currying adjunction:
\[ \varepsilon_{b, c} \colon [b, c] \times b \to c \]
suggests that $[b, c]$ embodies, in a sense, the inverse of multiplication. It plays a similar role as $c/b$ in:
\[ c/b \times b = c \]

In a typical categorical manner, we may ask the question: What if we replace the product with something else? The obvious thing, replacing it with a coproduct, doesn't work (so, we have no analog of subtraction). But maybe there are other well-behaved binary operations that have a right adjoint.

A natural setting for generalizing a product is a monoidal category with a tensor product $\otimes$ and a unit object $I$. If we have an adjunction:
\[ \cat C (a \otimes b, c) \cong \cat C (a, [b, c]) \]
we'll call the category \emph{closed monoidal}. In a typical categorical abuse of notation, unless it leads to confusion, we'll use the same symbol (a pair of square brackets)  for the monoidal internal hom as we did for the cartesian hom.

The definition of internal hom works well for a symmetric monoidal category. If the tensor product is not symmetric, the adjunction defines a \emph{left closed} monoidal category. The left internal hom is adjoint to the ``post-multiplication'' functor $(- \otimes b)$. The right-closed structure is defined as the right adjoint to the ``pre-multiplication'' functor $(b \otimes -)$. If both are defined than the category is called \index{bi-closed monoidal category}\emph{bi-closed}.


\subsection{Internal hom for Day convolution}

As an example, consider the symmetric monoidal structure in the category of co-presheaves with Day convolution:
\[ (F \star G) x = \int^{a, b} \cat C (a \otimes b, x) \times F a \times G b \]
We are looking for the adjunction:
\[ [\cat C, \Set] (F \star G, H) \cong  [\cat C, \Set] (F, [G, H]_{\text{Day}}) \]
The natural transformation in the left-hand side can be written as an end:
\[ \int_x \Set \big( \int^{a, b} \cat C (a \otimes b, x) \times F a \times G b, H x \big) \]
We can use co-continuity to pull out the coends:
\[ \int_{x, a, b} \Set \big( \cat C (a \otimes b, x) \times F a \times G b, H x \big) \]
We can then use the currying adjunction in $\Set$ (the square brackets stand for the internal hom in $\Set$, that is the exponential object):
\[ \int_{x, a, b} \Set \big( F a, [C (a \otimes b, x)  \times G b, H x] \big) \]
Finally, we use the continuity of the hom-set to move the two ends:
\[ \int_{a} \Set \big( F a, \int_{x, b} [C (a \otimes b, x)  \times G b, H x] \big) \]
We get that the right adjoint to Day convolution is given by:
\[ \big([G, H]_{\text{Day}}\big) a = \int_{x, y} \big[\cat C(a \otimes x, y), [G x, H y]\big] \cong \int_x [G x, H (a \otimes x)]\]
The last transformation is the application of the Yoneda lemma in $\Set$.
\begin{exercise}
Implement the internal hom for Day convolution in Haskell. Hint: Use a type alias.
\end{exercise}

\subsection{Powering and co-powering}

In $\Set$, the internal hom or the exponential is isomorphic to the external hom:
\[ \Set (A \times B, C) \cong \Set \big(A, \Set (B, C)\big) \]
The external hom in any category is always a set. We can therefore generalize this adjunction to the case where $B$ and $C$ are not sets but objects in some category $\cat C$. Such an adjunction defines the action of a set $A$ on an object $B$:
\[ \cat C (A \cdot b, c) \cong \Set \big( A, \cat C (b, c)\big) \]
or a \emph{co-power}.

You may think of this action as adding together (taking a coproduct of) $A$ copies of $b$. In particular, if $A$ is a two-element set $\mathbf 2$, we get:
\[ \cat C (\mathbf 2 \cdot b, c) \cong \Set \big( \mathbf 2, \cat C (b, c)\big) \cong \cat C(b, c) \times \cat C(b, c) \cong \cat C(b + b, c) \]
In other words, 
\[ \mathbf 2 \cdot b \cong b + b \]
In this sense a co-power defines multiplication as iterated addition. 

As expected, in $\Set$, the co-power decays to the cartesian product.
\[ \Set (A \cdot B, C) \cong \Set \big( A, \Set(B, C)\big) \cong \Set (A \times B, C) \]

Similarly, we can express powering as iterated multiplication. We use the same right-hand side, but this time we use the mapping-in to define the \emph{power}:
\[ \cat C (b, A \pitchfork c) \cong \Set  \big(A, \cat C(b, c)\big) \]
You may think of the power as multiplying together $A$ copies of $c$. Indeed, replacing $A$ with $\mathbf 2$ results in:
\[ \cat C (b, \mathbf 2 \pitchfork c) \cong \Set  \big(\mathbf 2, \cat C(b, c)\big) \cong \cat C(b, c) \times \cat C(b, c) \cong \cat C (b, c \times c)\]
In other words:
\[ \mathbf 2 \pitchfork c \cong c \times c \]
which is a fancy way of writing $c^2$.

In $\Set$, the power decays to the exponential, which is the same as the hom-set:
\[ A \pitchfork C \cong C^A \cong \Set (A, C) \]
This is the consequence of the symmetry of the product.
\[ \Set(B, A \pitchfork C) \cong \Set (A, \Set(B, C)) \cong \Set (A \times B, C) \]
\[ \cong \Set (B \times A, C) \cong \Set (B, \Set (A, C))\]

\section{Inverting a functor}

One side of category theory is discarding information by performing lossy transformations, the other is recovering the lost information. We've seen examples of making up for lost data with free functors,---the adjoints to forgetful functors. Kan extensions are another example. Both make up for data that is lost by a functor.

There are two reasons why a functor might not be invertible. One is that it may map multiple objects or arrows into a single object or arrow. In other words, it's not injective on objects or arrows. The other reason is that its image may not cover the whole target category. 

Consider for instance an adjunction $L \dashv R$. Suppose that $R$ collapses two object $c$ and $c'$ into a single object $d$
\begin{align*}
R c &= d \\
R c' &= d
\end{align*}
$L$ has no chance of undoing it. It can't map $d$ to both $c$ and $c'$. The best it can do is to map $d$ to an object that has arrows to both $c$ and $c'$ such that the counit of the adjunction can be made natural:
\[ \varepsilon \colon L \circ R \to Id \]
In components:
\[ \varepsilon_c \colon L d \to c \]
\[ \varepsilon_{c'} \colon L d \to c' \]
Moreover, if $R$ is not surjective on objects, the functor $L$ must somehow be defined on those objects of $\cat D$ that are not in the image of $R$. Again, naturality of the unit and counit constrain possible choices, as long as there are arrows connecting these objects to the image of $R$. 

Obviously, all these constraints mean that an adjunction can only be defined in very special cases. Kan extensions are even weaker than adjunctions. If adjoint functors work like inverses, Kan extensions work like fractions. 

This is best seen if we redraw the diagrams defining the counit and the unit of an adjunction. In the first diagram, $L$ seems to play the role of $1/R$. In the second diagram $R$ pretends to be $1/L$.

\[
 \begin{tikzcd} [row sep=1cm, column sep=1cm]
 \cat C
 \arrow[rr, "Id", "" {name=ID, below} ]
 \arrow[d, bend right, "R"']
 && \cat C
 \\
 \cat D
  \arrow[urr, bend right, "L"']
 \arrow[Rightarrow, "\varepsilon",  to=ID]
 \end{tikzcd}
 \qquad
 \begin{tikzcd} [row sep=1cm, column sep=1cm]
 \cat D
 \arrow[rr, "Id", "" {name=ID, below} ]
 \arrow[d, bend right, "L"']
 && \cat D
 \\
 \cat C
  \arrow[urr, bend right, "R"']
 \arrow[Rightarrow, "\eta",  from=ID]
 \end{tikzcd}
\]

The right Kan extension $\text{Ran}_P F$ and the left Kan extension $\text{Lan}_P F$ generalize these by replacing the identity functor with a functor $F \colon \cat C \to \cat D$. The Kan extensions then play the role of fractions $F/P$. Conceptually, they undo the action of $P$ and follow it with the action of $F$.

\[
 \begin{tikzcd} [row sep=1cm, column sep=1cm]
 \cat C
 \arrow[rr, "F", "" {name=ID, below} ]
 \arrow[d, bend right, "P"']
 && \cat D
 \\
 \cat B
  \arrow[urr, bend right, "\text{Ran}_P F"']
 \arrow[Rightarrow, "\varepsilon",  to=ID]
 \end{tikzcd}
 \qquad
 \begin{tikzcd} [row sep=1cm, column sep=1cm]
 \cat C
 \arrow[rr, "F", "" {name=ID, below} ]
 \arrow[d, bend right, "P"']
 && \cat D
 \\
 \cat B
  \arrow[urr, bend right, "\text{Lan}_P F"']
 \arrow[Rightarrow, "\eta",  from=ID]
 \end{tikzcd}
\]

Just like with adjunctions, the ``undoing'' is not complete. The composition $\text{Ran}_P F \circ P$ doesn't reproduce $F$; instead it's related to it through the natural transformation $\varepsilon$ called the counit. Similiarly, the composition $\text{Lan}_P F \circ P$ is related to $F$ through the unit $\eta$.

Notice that the more information $F$ discards, the easier it is for Kan extensions to ``invert'' the functor $P$. In as sense, it only has to invert $P$ ``modulo $F$''.

\section{Right Kan extension}



\[
 \begin{tikzcd} [row sep=1cm, column sep=1cm]
 \cat C
 \arrow[r, "F"]
 \arrow[d, "P"']
 & \cat D
 \\
 \cat B
  \arrow[ur, "\text{Ran}_P F"']
 \end{tikzcd}
\]

\[
 \begin{tikzcd} [row sep=1cm, column sep=1cm]
 \cat C
 \arrow[rr, "F", "" {name=ID, below} ]
 \arrow[d, bend right, "P"']
 && \cat D
 \\
 \cat B
  \arrow[urr, bend right, "\text{Ran}_P F"']
 \arrow[Rightarrow, "\varepsilon",  to=ID]
 \end{tikzcd}
\]



\[
\begin{tikzcd}[row sep=2cm, column sep=2cm]
\cat C  \ar[d, "P"', "" {name=P}]
            \ar[r, "F", ""  {name=F, below, near start, bend right}]
&
\cat D
\\
\cat B
    \ar[ur, bend left, "G", "" {name=G, below}]
    \ar[ur, bend right, "\text{Ran}_P F"', "" {name=Ran}]
\arrow[Rightarrow, "\sigma", from=G, to=Ran]
\end{tikzcd}
\]

\[ \cat C (a \otimes b, c) \cong \cat C (a, [b, c]) \]

\[ P^* G = G \circ P \]
\[ (G \circ P, F) \cong (G, \text{Ran}_P F) \]
 
 \[ (\cat C \times \cat C) (\Delta a , \langle b, c \rangle) \cong (a, b \times c) \]
 
 \[ (\text{Ran}_P F) e \cong \int_c \cat B (e, P c) \pitchfork F c \]
 
  \[ (\text{Ran}_P F) e \cong \int_c \Set \big( \Set (e, P c), F c\big) \]

 \begin{haskell}
 newtype Ran p f e = Ran (forall c. (e -> p c) -> f c)
 \end{haskell}

\section{Left Kan extension}
\[ (\text{Lan}_P F , G) \cong (F, G \circ P) \]
\[
\begin{tikzcd}[row sep=2cm, column sep=2cm]
\cat C  \ar[d, "P"', "" {name=P}]
            \ar[r, "F", ""  {name=F, below, near start, bend right}]
&
\cat D
\\
\cat B
    \ar[ur, bend left, "G", "" {name=G, below}]
    \ar[ur, bend right, "\text{Lan}_P F"', "" {name=Lan}]
\arrow[Rightarrow, "", from=Lan, to=G]
\end{tikzcd}
\]



\[ (\text{Lan}_P F) e \cong \int^{c} \cat B(P c, e) \cdot F c \]

\[ (\text{Lan}_P F) e \cong \int^{c} \Set (P c, e) \times F c \]

 \begin{haskell}
 Lan p f e = exists c. (p c -> e, f c)
 \end{haskell}

 \begin{haskell}
 data Lan p f e where
   Lan :: (p c -> e) -> f c -> Lan p f e
 \end{haskell}
\begin{itemize}

\item Kan extensions as the adjoints to functor pre-composition
\item (co-) end representation

\item (co-) limits as Kan extensions

\item adjunctions as Kan extensions

\item pointwise tensor product

\item Day convolution as Kan extension

\item Kan lift as an adjoint to functor post-composition
\end{itemize}

\subsection{Notes}

\section{Useful Formulas}
\begin{itemize}
\item Co-power:
\[ \cat C (A \cdot b, c) \cong \Set \big( A, \cat C (b, c)\big) \]
\item Power:
\[ \cat C (b, A \pitchfork c) \cong \Set  \big(A, \cat C(b, c)\big) \]
\item Right Kan extension:
 \[ (\text{Ran}_P F) e \cong \int_c \cat B (e, P c) \pitchfork F c \]
\item Right Kan extension in $\Set$:
  \[ (\text{Ran}_P F) e \cong \int_c \Set \big( \Set (e, P c), F c\big) \]
\item Left Kan extension:
\[ (\text{Lan}_P F) e \cong \int^{c} \cat B(P c, e) \cdot F c \]
\item Left Kan extension in $\Set$:
\[ (\text{Lan}_P F) e \cong \int^{c} \Set (P c, e) \times F c \]

\end{itemize}

\end{document}
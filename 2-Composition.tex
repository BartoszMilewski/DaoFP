\documentclass[DaoFP]{subfiles}
\begin{document}
\setcounter{chapter}{1}
\chapter{Composition}

\section{Composition}

Programming is about composition. Paraphrasing Wittgenstein, one could say: ``Of that which cannot be decomposed one should not speak.'' This is not a prohibition, it's a statement of fact. The process of studying, understanding, and describing is the same as the process of decomposing; and our language reflects this. 

The reason we have built the vocabulary of objects and arrows is precisely to express the idea of composition.  

Given an arrow $f$ from $a$ to $b$ and an arrow $g$ from $b$ to $c$, their composition is an arrow that goes directly from $a$ to $c$. In other words, if there are two arrows, the target of one being the same as the source of the other, we can always compose them to get a third arrow.

\[
 \begin{tikzcd}
 a
 \arrow[rr, bend left, "h"]
 \arrow[r, "f"']
 & b
 \arrow[r, "g"']
& c
 \end{tikzcd}
\]

In math we denote composition using a little circle
\[h = g \circ f\]
We read this: ``$h$ is equal to $g$ after $f$.'' The choice of the word ``after'' suggests temporal ordering of actions, which in most cases is a useful intuition.

The order of composition might seem backward, but this is because we think of functions as taking arguments on the right.
In Haskell we replace the circle with a dot:
\begin{haskell}
h = g . f
\end{haskell}
This is every program in a nutshell. In order to accomplish \hask{h}, we decompose it into simpler problems, \hask{f} and \hask{g}. These, in turn, can be decomposed further, and so on.

Now suppose that we were able to decompose $g$ itself into $j \circ k$. We have
\[h = (j \circ k) \circ f\]
We want this decomposition to be the same as
\[h = j \circ (k \circ f)\]
We want to be able to say that we have decomposed $h$ into three simpler problems
\[h =  j \circ k \circ f\]
and not have to keep track which decomposition came first. This is called \emph{associativity} of composition, and we will assume it from now on.

Composition is the source of two mappings of arrows called pre-composition and post-composition. 

When you \index{post-composition}\emph{post-compose} an arrow $h$ with an arrow $f$, it produces the arrow $f \circ h$ (the arrow $f$ is applied \emph{after} the arrow $h$). Of course, you can post-compose $h$ only with arrows whose source is the target of $h$. Post-composition by $f$ is written as $(f \circ -)$, leaving a hole for $h$. As Lao Tzu would say, ``Usefulness of post-composition comes from what is not there.''

Thus an arrow $f \colon a \to b$ induces a mapping of arrows $(f \circ -)$ that maps arrows which are probing $a$ to arrows which are probing $b$. 

\[
 \begin{tikzcd}
 \node (a1) at (0, 2) {};
 \node (a2) at (-0.5, 2) {};
 \node (a3) at (0.5, 2) {};
 \node (aa) at (0.5, 1) {};
 \node(a) at (0, 0) {a};
 \draw[->, red] (a1) -- (a);
 \draw[->] (a2) -- (a);
 \draw[->, blue] (a3) -- (a);
 \node (b1) at (3+0, 2) {};
 \node (b2) at (3-0.5, 2) {};
 \node (b3) at (3+0.5, 2) {};
 \node (bb) at (2.5, 1) {};
 \node(b) at (3, 0) {b};
 \draw[->, red] (b1) -- (b);
 \draw[->] (b2) -- (b);
 \draw[->, blue] (b3) -- (b);
 \draw[->] (a) -- node[below]{f} (b);
 \draw[->, dashed] (aa) -- node[above]{(f \circ -)} (bb);
  \end{tikzcd}
\]
Since objects have no internal structure, when we say that $f$ transforms $a$ to $b$, this is exactly what we mean. 

Post-composition lets us shift focus from one object to another.

Dually, you can \index{pre-composition}\emph{pre-compose} by $f$, or apply $(- \circ f)$ to arrows originating in $b$ to map them to arrows originating in $a$ (notice the change of direction). 

\[
 \begin{tikzcd}
 \node (a1) at (0, -2) {};
 \node (a2) at (-0.5, -2) {};
 \node (a3) at (0.5, -2) {};
 \node (aa) at (0.5, -1) {};
 \node(a) at (0, 0) {a};
 \draw[<-, red] (a1) -- (a);
 \draw[<-] (a2) -- (a);
 \draw[<-, blue] (a3) -- (a);
 \node (b1) at (3+0, -2) {};
 \node (b2) at (3-0.5, -2) {};
 \node (b3) at (3+0.5, -2) {};
 \node (bb) at (2.5, -1) {};
 \node(b) at (3, 0) {b};
 \draw[<-, red] (b1) -- (b);
 \draw[<-] (b2) -- (b);
 \draw[<-, blue] (b3) -- (b);
 \draw[->] (a) -- node[above]{f} (b);
 \draw[->, dashed] (bb) -- node[below]{(- \circ f)} (aa);
  \end{tikzcd}
\]

Pre-composition lets us shift the perspective from observer $b$ to observer $a$.

Pre- and post-composition are mappings of arrows. Since arrows form sets, these are \emph{functions} between sets. 

Another way of looking at pre- and post-composition is that they are the result of partial application of the two-hole composition operator $(- \circ -)$, in which we can pre-fill one hole or the other with a fixed arrow.

In programming, an outgoing arrow is interpreted as extracting data from its source. An incoming arrow is interpreted as producing or constructing the target. Outgoing arrows define the interface, incoming arrows define the constructors.

For Haskell programmers, here's the implementation of post-composition as a higher-order function:
\begin{haskell}
postCompWith :: (a -> b) -> (x -> a) -> (x -> b)
postCompWith f = \h -> f . h
\end{haskell}
And similarly for pre-composition:
\begin{haskell}
preCompWith :: (a -> b) -> (b -> x) -> (a -> x)
preCompWith f = \h -> h . f
\end{haskell}

Do the following exercises to convince yourself that shifts in focus and perspective are composable.
\begin{exercise}\label{ex-yoneda-composition}
Suppose that you have two arrows, $f \colon a \to b$ and $g \colon b \to c$. Their composition $g \circ f$ induces a mapping of arrows $((g \circ f) \circ -)$. Show that the result is the same if you first apply $(f \circ -)$ and follow it by $(g \circ -)$. Symbolically:
\[((g \circ f) \circ -) = (g \circ -) \circ (f \circ -)\]

Hint: Pick an arbitrary object $x$ and an arrow $h \colon x \to a$ and see if you get the same result. Note that $\circ$ is overloaded here. On the right, it means regular function composition when put between two post-compositions.
\end{exercise}

\begin{exercise}
Convince yourself that the composition from the previous exercise is associative. Hint: Start with three composable arrows.
\end{exercise}

\begin{exercise}
Show that pre-composition $(- \circ f)$ is composable, but the order of composition is reversed:
\[(- \circ (g \circ f)) = (- \circ f) \circ (- \circ g) \]
\end{exercise}

\section{Function application}

We are ready to write our first program. There is a saying: ``A journey of a thousand miles begins with a single step.'' Consider a journey from $1$ to $b$. Our single step can be an arrow from the terminal object $1$ to some $a$. It's an element of $a$. We can write it as:
\[1 \xrightarrow x a \]
The rest of the journey is the arrow:
\[a \xrightarrow f b\]
These two arrows are composable (they share the object $a$ in the middle) and their composition is the arrow $y$ from $1$ to $b$. In other words, $y$ is an \emph{element} of $b$:

\[
 \begin{tikzcd}
 1
 \arrow[rr, bend left, "y"]
 \arrow[r, "x"']
 & a
 \arrow[r, "f"']
& b
 \end{tikzcd}
\]
We can write it as:
\[y = f \circ x \]

We used $f$ to map an \emph{element} of $a$ to an \emph{element} of $b$. Since this is something we do quite often, we call it the \emph{application} of a function $f$ to $x$, and use the shorthand notation
\[y = f x \]
Let's translate it to Haskell. We start with an element $x$ of $a$ (a shorthand for \hask{x :: ()-> a})
\begin{haskell}
x :: a
\end{haskell}
We declare a function $f$ as an element of the ``object of arrows'' from $a$ to $b$
\begin{haskell}
f :: a -> b
\end{haskell}
with the understanding (which will be elaborated upon later) that it corresponds to an arrow from \hask{a} to \hask{b}. The result is an element of $b$
\begin{haskell}
y :: b
\end{haskell}
and it is defined as
\begin{haskell}
y = f x
\end{haskell}
We call this the application of a function to an argument, but we were able to express it purely in terms of function composition. (Note: In other programming languages function application requires the use of parentheses, e.g., \hask{y = f(x)}.)

\section{Identity}

You may think of arrows as representing change: object $a$ becomes object $b$. An arrow that loops back represents a change in an object itself. But change has its dual: lack of change, inaction or, as Lao Tzu would say \emph{wu wei}. 

Every object has a special arrow called the identity, which leaves the object unchanged. It means that, when you compose this arrow with any other arrow, either incoming or outgoing, you get that other arrow back. Thought of as an action, the identity arrow does nothing and takes no time.

An identity arrow on the object $a$ is called $id_a$. So if we have an arrow $f \colon a \to b$, we can compose it with identities on either side

\[id_b \circ f = f = f \circ id_a \]
or, pictorially:
\[
 \begin{tikzcd}
 a
 \arrow[loop, "id_a"']
 \arrow[r, "f"]
 & b
 \arrow[loop, "id_b"']
 \end{tikzcd}
\]

We can easily check what an identity does to elements. Let's take an element $x \colon 1 \to a$ and compose it with $id_a$. The result is:
\[id_a \circ x = x\]
which means that identity leaves elements unchanged.

In Haskell, we use the same name \hask{id} for all identity functions (we don't subscript it with the type it's acting on). The above equation, which specifies the action of $id$ on elements, translates directly to:
\begin{haskell}
id x = x
\end{haskell}
and it becomes the definition of the function \hask{id}. 

We've seen before that both the initial object and the terminal object have unique arrows circling back to them. Now we are saying that every object has an identity arrow circling back to it. Remember what we said about uniqueness: If you can find two of those, then they must be equal. We must conclude that these unique looping arrows we talked about must be the identity arrows. We can now label these diagrams:

\[
 \begin{tikzcd}
 \hask{Void}
 \arrow[loop, "id"']
 \end{tikzcd}
 \begin{tikzcd}
 \hask{()}
 \arrow[loop, "id"']
 \end{tikzcd}
\]

In logic, identity arrow translates to a tautology. It's a trivial proof that, ``if $a$ is true then $a$ is true.'' It's also called the \emph{identity rule}.


If identity does nothing then why do we care about it? Imagine going on a trip, composing a few arrows, and finding yourself back at the starting point. The question is: Have you done anything, or have you wasted your time? The only way to answer this question is to compare your path to the identity arrow. 

Some round trips bring change, others don't.

More importantly, identity arrows will allow us to compare objects. They are an integral part of the definition of an isomorphism.

\begin{exercise}\label{ex-yoneda-identity}
What does $(id_a \circ -)$ do to arrows terminating in $a$? What does $(- \circ id_a)$ do to arrows originating from $a$?
\end{exercise}



\end{document}
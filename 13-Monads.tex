\documentclass[DaoFP]{subfiles}
\begin{document}
\setcounter{chapter}{12}

\chapter{Monads}

What does a wheel, a clay pot, and a wooden house have in common? They are all useful because of the emptiness in their center. 

Lao Tzu says: ``The value comes from what is there, but the use comes from what is not there.''

What does the \hask{Maybe} functor, the list functor, and the reader functor have in common? They all have emptiness in their center. 

When monads are explained in the context of programming, it's hard to see the common pattern when you focus on the functors. 

\subsection{notes}


\begin{exercise}
\end{exercise}

\begin{haskell}
\end{haskell}

\[
 \begin{tikzcd}
  \end{tikzcd}
\]

\[   \mathbf{Set} \]
\[   \mathcal{C} \]

\end{document}
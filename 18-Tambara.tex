\documentclass[DaoFP]{subfiles}
\begin{document}
\setcounter{chapter}{17}

\chapter{Tambara Modules}

It's not often that an obscure corner of category theory gains sudden prominence in programming. Tambara modules got a new lease on life in their application to profunctor optics. They provide a clever solution to the problem of composing optics. We've seen that, in the case of lenses, the getters compose nicely using function composition, but the composition of setters involves some shenanigans. The existential representation doesn't help much. The profunctor representation, on the other hand, makes composition a snap. 

The situation is somewhat analogous to the problem of composing geometric transformations in graphics programming. For instance, if you try to compose two rotations around two different axes, the formula for the new axis and the angle is quite complicated. But if you represent rotations as matrices, you can use matrix multiplication; or, if you represent them as quaternions, you can use quaternion multiplication. Profunctor representation lets you compose optics using straightforward function composition.

\section{Tannakian Reconstruction}

\subsection{Monoids and their Representations}

The theory or representations is a science in itself. Here, we'll approach it from the categorical perspective. Instead of groups, we'll consider monoids. A monoid can be defined as a special object in a monoidal category, but it can also be thought of as a single-object category $\cat M$. If we call this object $*$, the hom-set $\cat M( *, *)$ contains all the information we need. 

Monoidal product is simply the composition of morphisms. By the laws of a category, it's associative and unital---the identity morphism serving as the monoidal unit.

In this sense, every single-object category is automatically a monoid and all monoids can be made into single-object categories. 

For instance, a monoid of whole numbers with addition can be thought of as a category with a single object $*$ and a morphism for every number. To compose two such morphisms, you add their numbers, as in the example below:
\[
 \begin{tikzcd}
  *
  \arrow[r, bend left, "2"]
  \arrow[rr, bend right, "5"']
 & *
 \arrow[r, bend left, "3"]
 & *
  \end{tikzcd}
\]
The morphism corresponding to zero is automatically the identity morphism. 

We can represent a monoid on a set. Such a representation is a functor $F$ from $\cat M$ to $\Set$. Such a functor maps the single object $*$ to some set $S$, and it maps the hom-set $\cat M(*, *)$ to a set of functions $S \to S$. By functor laws, it maps identity to identity and composition to composition, so it preserves the structure of the monoid. 

If the functor is fully faithful, its image contains exactly the same information as the monoid and nothing more. But, in general, functors cheat. The hom-set $\Set (S, S)$ may contain some other functions that are not in the image of $\cat M(*, *)$; and multiple morphisms can be mapped to a single function. 

In the extreme, the whole hom-set $\cat M(*, *)$ may be mapped to the identity morphism $id_S$. So, just by looking at the set $S$---the image of $*$ under the functor $F$---we cannot dream of reconstructing the original monoid.

Not all is lost, though, if we are allowed to look at all the representations of a given monoid simultaneously. Such representations form a functor category $[\cat M, \Set]$, a.k.a. the co-presheaf category. Arrows in this category are natural transformations. 

Since the source category $\cat M$ contains only one object, naturality conditions take a particularly simple form. A natural transformation $\alpha \colon F \to G$ has only one component, a function $\alpha \colon F * \to G *$. Given a morphism $m \colon * \to *$, the naturality square reads:

\[
 \begin{tikzcd}
 F *
 \arrow[r, "\alpha"]
 \arrow[d, "F m"]
 & G *
  \arrow[d, "G m"]
\\
 F *
 \arrow[r, "\alpha"]
 & G *
 \end{tikzcd}
\]
It's a relationship between three functions acting on two sets:
\[
 \begin{tikzcd}
 F *
  \arrow[loop, "F m"']
  \arrow[rr, bend right, "\alpha"']
 && G *
  \arrow[loop, "G m"']
  \end{tikzcd}
\]
The naturality condition tells us that:
\[ G m \circ \alpha = \alpha \circ F m \]

In other words, if you pick an element $x \in F *$, you can map it to $G *$ using $\alpha$ and then apply the transformation corresponding to $m$; or you can first apply the thansformation $F m$ and then map the result using $\alpha$. The result is the same in both cases.

Such functions are called \index{equivariant function}\emph{equivariant}. We often call $F m$ the \emph{action} of $m$ on the set $F *$. An equivariant function maps an action on one set to its corresponding action on another set. 

\subsection{Tannakian reconstruction of a monoid}

How much information do we need to reconstruct a monoid from its representations? Just looking at the sets is definitely not enough, since any monoid can be represented on any set. But if we include structure-preserving functions between these sets, we might have a chance. 

Consider a set of functions $\Set(F *, F *)$ for a given functor $F \colon \cat M \to \Set$. At least some of these are the actions of the monoid, $F m$. If we look at the ``neighboring'' set $G *$ and its set of functions  $\Set(G *, G *)$ we'll find there the corresponding actions $G m$. An equivariant function, that is a natural transformation in $[\cat M, \Set]$, will map these actions. It will map other morphisms as well.

Imagine creating a gigantic tuple by taking one function from each of the sets $\Set (F *, F *)$, for all functors $F\colon \cat M \to \Set$. Moreover, we want the elements of the tuple to be correlated. If we pick $g \in  \Set (G*, G *)$ and  $h \in  \Set (H*,  H*)$ and there is a natural transformation (equivariant function) $\alpha$ between the two functors $G$ and $H$, we want the two function to be related:
\[ \alpha \circ g = h \circ \alpha \]
or, pictorially:
\[
 \begin{tikzcd}
 G *
  \arrow[loop, "g"']
  \arrow[rr, bend right, "\alpha"']
 && H *
  \arrow[loop, "h"']
  \end{tikzcd}
\]
Notice that this will guarantee that actions are mapped to corresponding actions if, for instance, $g = G m$ and $h = H m$. 

Such tuples are exactly the elements of the end:
\[ \int_F \Set(F *, F *) \]
whose wedge condition provides the constraints we are looking for. 

\[
 \begin{tikzcd}
 & \int_F \Set (F *, F *)
 \arrow[dl, "\pi_G"']
 \arrow[dr, "\pi_H"]
 \\
 \Set(G *, G *)
 \arrow[dr, "{\Set(id, \alpha)}"']
 && \Set (H *, H *)
  \arrow[dl, "{\Set (\alpha, id)}"]
\\
 & \Set (G *, H *)
 \end{tikzcd}
\]

Notice that this is the end over the whole functor category $[\cat M, \Set ]$, so the wedge condition relates the entries that are connected by natural transformations. In this case, natural transformations are equivariant functions. The profunctor under the end is given by:
\[ P \langle G, H \rangle = \Set (G *, H *) \]
It is a functor:
\[ P \colon [\cat M, \Set]^{op} \times [\cat M, \Set] \to \Set \]
Its action on morphisms (natural transformations) is:
\[ P \langle \alpha, \beta \rangle = \beta \circ - \circ \alpha \]
If we pick $f$ to be the element of $\Set (G*, G*)$ and $g$ to be the element of $\Set (H*, H*)$, the wedge condition indeed becomes:
\[ \alpha \circ g = h \circ \alpha \]

The Tannakian reconstruction theorem, in this case, tells us that:
\[ \int_F \Set(F *, F *) \cong \cat M (*, *) \]
In other words, we can recover the monoid from its representations.

\subsection{Proof of Tannakian reconstruction}

Monoid reconstruction is a special case of a more general theorem in which the single-object category is replaced with a regular category. As in the monoid case, we'll reconstruct the hom-set, only this time it will be a general hom-set. We'll prove the formula:
\[ \int_{F \colon [\cat C, \Set]} \Set (F a, F b) \cong \cat C(a, b) \]
The trick is to use the Yoneda lemma to represent the action of $F$:
\[ F a \cong [\cat C, \Set] ( \cat C (a ,-), F) \]
and the same for $F b$. We get:
\[ \int_{F \colon [\cat C, \Set]} \Set ([\cat C, \Set] ( \cat C (a ,-), F), [\cat C, \Set] ( \cat C (b ,-), F)) \]
The two sets of natural transformations here are hom-sets in $[\cat C, \Set]$. 

Recall the corollary to the Yoneda lemma that works for any category $\cat A$:
\[ [\cat A, \Set] (\cat A (x, -), \cat A (y, -)) \cong \cat A (y, x) \]
We can write it using an end:
\[ \int_{z \colon \cat C} \Set (\cat A (x, z), \cat A (y, z)) \cong \cat A (y, x) \]
In particular, we can replace $\cat A$ with the functor category $[\cat C, \Set]$. We get:
\[ \int_{F \colon [\cat C, \Set]} \Set ([\cat C, \Set] ( \cat C (a ,-), F), [\cat C, \Set] ( \cat C (b ,-), F)) \cong [\cat C, \Set](\cat C (b ,-), \cat C (a ,-))\]
We can then apply the Yoneda lemma again to the right hand side to get:
\[ \cat C (a, b) \]
which is exactly the sought after result.

\subsection{Tannakian reconstruction in Haskell}

We can immediately translate this result to Haskell. We replace the end by \hask{forall}. The left hand side becomes:
\begin{haskell}
forall f. Functor f => f a -> f b
\end{haskell}
and the right hand side is the function type \hask{a->b}. 

We've seen polymorphic functions before: they were functions defined for all types, or sometimes for classes of types. Here we have a function that is defined for all functors. It says: give me a functorful of \hask{a}'s and I'll produce a functorful of \hask{b}'s---no matter what functor you use. The only way this can be implemented (using parametric polymorphism) is if this function has secretly captured a function of the type \hask{a->b} and is applying it using \hask{fmap}. 

Indeed, one direction of the isomorphism is just that: capturing a function and \hask{fmap}ping it over the argument:
\begin{haskell}
toRep :: (a -> b) -> (forall f. Functor f => f a -> f b)
toRep g fa = fmap g fa
\end{haskell}
The other direction uses the identity trick:
\begin{haskell}
fromRep :: (forall f. Functor f => f a -> f b) -> (a -> b)
fromRep g a = unId (g (Id a))
\end{haskell}
where the identity functor is defined as:
\begin{haskell}
data Id a = Id a 
  
unId :: Id a -> a
unId (Id a) = a

instance Functor Id where
  fmap g (Id a) = Id (g a)
\end{haskell}

This kind of reconstruction might seem trivial and pointless. Why would anyone want to replace function type \hask{a->b} with a much more complicated type:
\begin{haskell}
type Getter a b = forall f. Functor f => f a -> f b
\end{haskell}
It's instructive, though, to think of \hask{a->b} as the precursor of all optics. It's a lens that focuses on the $b$ part of $a$. It tells us that $a$ contains enough information, in one form or another, to construct an $b$. It's a ``getter'' or an ``accessor.'' 

Obviously, functions compose. What's interesting though is that functor representations also compose, and they compose using simple function composition:
\begin{haskell}
boolToStrGetter :: Getter Bool String
boolToStrGetter = toRep (show) . toRep (bool 0 1)
\end{haskell}
Other optics don't compose so easily, but their functor (and profunctor) representation do. 
\subsection{Pointed getter}

Here's a little toy example in Haskell that illustrates the need for more interesting reconstructions. It's an optic that can either act as a getter or it can return a default value.
\begin{haskell}
data PtdGetter s t = Pt t | Fun (s -> t)
\end{haskell}
We can apply this getter to a source value and get a result:
\begin{haskell}
apply :: PtdGetter s t -> s -> t
apply (Pt t) _ = t
apply (Fun g) s = g s
\end{haskell}

These getters compose, but their composition is non-trivial:
\begin{haskell}
composePG :: PtdGetter x t -> PtdGetter s x -> PtdGetter s t
composePG (Pt t) _ = Pt t
composePG (Fun g) (Pt x) = Pt (g x)
composePG (Fun g) (Fun g') = Fun (g . g')
\end{haskell}
The composition is associative, and there is an identity getter:
\begin{haskell}
idPG :: PtdGetter a a
idPG = Fun id
\end{haskell}
so we do have a category in which pointed getters form hom-sets. 

The functor representation for this toy optic exists, but we have to restrict the type of functors over which we take the end. Here's the definition of a class of \hask{Pointed} functors:
\begin{haskell}
class Functor f => Pointed f where
  eta :: a -> f a
\end{haskell}
Our \hask{PointedGetter} is represented by the following Tannakian-like formula:
\begin{haskell}
type PtdGetterF s t = forall f. Pointed f => f s -> f t
\end{haskell}
This time we are defining a function that is polymorphic not over all functors but over a restricted class of \hask{Pointed} functors.

As before, we can apply this optic to retrieve the target. The trick is to encapsulate the source in the identity functor:
\begin{haskell}
applyF :: PtdGetterF s t -> s -> t
applyF g = unId . g . Id
\end{haskell}
Indeed, the identity functor is pointed:
\begin{haskell}
instance Pointed Id where
  eta = Id
\end{haskell}

The equivalence of the two formulations is witnessed by this pair of functions:
\begin{haskell}
toPGF :: PtdGetter s t -> PtdGetterF s t
toPGF (Pt t) = \_ -> eta t
toPGF (Fun g) = fmap g

fromPGF :: PtdGetterF s t -> PtdGetter s t
fromPGF g = Fun (unId . g . Id)
\end{haskell}
This time, however, the functor representation has definite advantage over the original: a composition of two \hask{PtdGetterF} optics is just function composition.
\begin{exercise}
Define two composable \hask{PtdGetter} optics---for instance, one going from a pair \hask{(Int, Bool)} to \hask{Int} and another from \hask{Int} to \hask{String}. Compose them first using \hask{composePG}, then convert them to the functor representation, and compose them using function composition.
\end{exercise}


\subsection{Tannakian reconstruction with adjunction}

In the toy example, we performed the reconstruction over a category of functors that were equipped with additional structure. In category theory we would describe pointed functors as endofunctors $P \colon \Set \to \Set$ equipped with natural transformations:
\[ \eta \colon Id \to P \]
They form their own category, let's call it $\mathbf{Ptd}$,  with morphisms that are natural transformations that preserve the structure. Such a transformation $\alpha \colon (P, \eta) \to (P', \eta')$ must make the following triangle commute:
\[
 \begin{tikzcd}
 & a
 \arrow[dl, "\eta"']
 \arrow[dr, "\eta'"]
 \\
 P a
 \arrow[rr, "\alpha_a"]
 && P' a
 \end{tikzcd}
\]

Every pointed functor is a functor---a statement that is formalized by saying that there is a forgetful functor $U \colon\mathbf{Ptd} \to [\Set, \Set]$, which forgets the additional structure. Notice that this is a functor between functor categories. 

This functor has a left adjoint. Any endofunctor $P \colon \Set \to \Set$ can be freely made into a pointed functor using the coproduct:
\[ (F Q) a = a + Q a \]
together with a natural transformation:
\[ \eta_a = \text{Left} \]

The trick in generalizing the Tannakian reconstruction is to define the end over a specialized functor category $\cat T$, 
 but applying the forgetful functor to its functors. We assume that we have the free/forgetful adjunction $F \dashv U$:
\[ \cat T (F Q, P) \cong  [\cat C, \Set] ( Q, U P )\]

Our starting point is the end:
\[ \int_{P \colon \cat T} \Set \big((U P) a, (U P) b\big) \]
The mapping $\cat T \to \Set$ given by:
\[ P \mapsto (U P) a \]
is sometimes called a fiber functor, so the end formula can be interpreted as a set of natural transformations between two fiber functors. 

As we did before, we first apply the Yoneda lemma to get:
\[ \int_{P \colon \cat T} \Set \Big([\cat C, \Set] \big( \cat C (a ,-), U P\big), [\cat C, \Set] \big( \cat C (b ,-), U P\big)\Big) \]
We can now use the adjunction:
\[ \int_{P \colon \cat T} \Set \Big(\cat T \big( F \cat C (a ,-), P\big), \cat T \big( F \cat C (b ,-), P\big)\Big) \]
We end up with a mapping between two natural transformations in the functor category $\cat T$. We can simplify it using the corollary to the Yoneda lemma:
\[ \cat T\big( F \cat C (b ,-), F \cat C (a ,-) \big) \]
We apply the adjunction once more:
\[ \cat T\big( \cat C (b ,-), (U\circ F) \cat C (a ,-) \big) \]
and the Yoneda lemma again:
\[ \big( (U\circ F) \cat C (a ,-) \big) b \]
The final observation is that the compostion $U \circ F$ of adjoint functors is a monad in the functor category. Let's call this monad $\Phi$. The result is the following identity that will serve as the foundation for profunctor optics:
\[ \int_{P \colon \cat T} \Set \big((U P) a, (U P) b\big) \cong \big( \Phi \cat C (a ,-) \big) b \]
The right-hand side is the action of the monad $\Phi = U \circ F$ on the representable functor $\cat C (a, -)$ evaluated at $b$. 

In our toy example, the monad $\Phi$ is given by:
\[ (\Phi Q) b = b + Q b \]
which is just the action of the free functor $F$ followed by forgetting the $\eta$. Replacing $Q$ with the representable $\cat C (a, -)$ we get:
\[ b + \cat C (a, b) \]
In Haskell, this translates directly to our \hask{PtdGetter}.

The bottom line is that we were able to reconstruct a hom-set in a category of simple optics from the category of functors with some additional structure. 

\section{Profunctor Lenses}

Our goal is to find a functor representation for other optics. We've seen before that, for instance, type-changing lenses can be seen as hom-sets in the $\mathbf{Lens}$ category. The objects in $\mathbf{Lens}$ are pairs of objects from some category $\cat C$, and a hom-set from one such pair $\langle s, t \rangle$ to another  $\langle a, b \rangle$ is given by the coend formula:
\[ \mathcal{L}\langle s, t\rangle \langle a, b \rangle = \int^{c} \mathcal{C}(s, c \times a) \times  \mathcal{C}(c \times b, t) \]
Notice that the pair of hom-sets in this formula can be seen as a single hom-set in the product category $\cat C^{op} \times \cat C$:
\[  \mathcal{L}\langle s, t\rangle \langle a, b \rangle =  \int^{c} (\cat C^{op} \times \cat C )(c \bullet \langle a, b \rangle, \langle s, t \rangle)  \]
where we define the action of $c$ on a pair $\langle a, b \rangle$ as:
\[ c \bullet \langle a, b \rangle = \langle c \times a, c \times b \rangle \]

This suggests that, to represent such optics, we should be looking at co-presheaves on the category $\cat C^{op} \times \cat C$, that is, we should be considering profunctor representations. 

\subsection{Iso}
As a quick check of this idea, let's apply our reconstruction formula to the simple case of $\cat T = \cat C^{op} \times \cat C$. In that case we don't need to use the forgetful functors, or the monad $\Phi$, and we get:
\[  \mathcal{O}\langle s, t\rangle \langle a, b \rangle =\int_{P \colon \cat C^{op} \times \cat C} \Set \big(P \langle a, b\rangle, P \langle s, t\rangle \big) \cong \big( (\cat C^{op} \times \cat C) (\langle a, b\rangle ,-) \big) \langle s, t\rangle \]
The right hans side evaluates to:
\[ (\cat C^{op} \times \cat C) (\langle a, b\rangle , \langle s, t\rangle) = \cat C (s, a) \times \cat C (b, t) \]

This optic is known in Haskell as \hask{Iso}:
\begin{haskell}
type Iso s t a b = (s -> a, b -> t)
\end{haskell}
and it, indeed, has a profunctor representation:
\begin{haskell}
type Iso s t a b = forall p. Profunctor p => p a b -> p s t
\end{haskell}
Given a pair of functions it's easy to construct this profunctor-polymorphic function:
\begin{haskell}
toIsoP :: (s -> a, b -> t) -> Iso s t a b
toIsoP (f, g) = dimap f g
\end{haskell}
This is simply saying that any profunctor can be used to lift a pair of functions. 

Conversely, we may ask the question: How can a function map the set $P \langle a, b \rangle$ to the set $P \langle s, t \rangle$ for \emph{every} profunctor imaginable? The only thing this function knows about the profunctor is that it can lift a pair of functions. Therefore it must either contain or be able to produce these two functions. 

\begin{exercise}
Implement the function:
\begin{haskell}
fromIsoP :: Iso s t a b -> (s -> a, b -> t)
\end{haskell}
Hint: Use the fact that a pair of identity functions can be used to construct the following profunctor:
\begin{haskell}
data Adapter a b s t = Ad (s -> a, b -> t)
\end{haskell}
\end{exercise}

\subsection{Profunctors and lenses}

Let's try to apply the same logic to lenses. We have to find a class of profunctors to plug into our profunctor representation. Let's assume that the forgetful functor $U$ only forgets additional structure but doesn't change the sets, so the set $P \langle a, b \rangle$ is the same as the set $(U P) \langle a, b \rangle$. 

Let's start with the existential representation. We are given an object $c$ and a pair of functions:
\[  \langle f, g \rangle \colon \cat C(s, c \times a) \times \cat C(c \times b, t) \]
We want to build a profunctor representation, so we have to be able to map the set $P \langle a, b \rangle$ to the set $P \langle s, t \rangle$. We could get $P \langle s, t \rangle$ by lifting these two functions, but we would have to start from $P \langle c \times a, c \times b \rangle$. Indeed:
\[ P \langle f, g \rangle \colon P \langle c \times a, c \times b \rangle \to P \langle s, t \rangle \]
What we are missing is the mapping:
\[ P \langle a, b \rangle \to P \langle c \times a, c \times b \rangle \]
And this is exactly the additional structure we shall demand from our profunctor class. 

\subsection{Tambara module}

A profunctor $P$ equipped with the family of transformations:
\[ \alpha_{\langle a, b\rangle, c} \colon P \langle a, b \rangle \to P \langle c \times a, c \times b \rangle \]
is called a \emph{Tambara module}. 

We want this family to be a natural in $a$ and $b$, but what should we demand from $c$? The problem with $c$ is that it appears twice, once in a contravariant, and once in a covariant position. So, if we want to interact nicely with arrows like $h \colon c \to c'$, we have to modify the naturality condition. We may consider a more general profunctor $P \langle c' \times a, c \times b \rangle$ and treat $\alpha$ as producing its diagonal elements, ones for which $c'$ is the same as $c$. 

A transformation $\alpha$ between diagonal parts of two profunctors $P$ and $Q$ is called a \index{dinatural transformation}\emph{dinatural transformation} (\emph{di}-agonally natural) if the following diagram commutes for any $f \colon c \to c'$:

\[
 \begin{tikzcd}
 & P \langle c', c \rangle
 \arrow[dl, "{P \langle f, c \rangle}"']
 \arrow[dr, "{P \langle c', f \rangle}"]
 \\
 P \langle c, c \rangle 
  \arrow[d, "\alpha_c"']
 && P \langle c', c' \rangle
 \arrow[d, "\alpha_{c'}"]
 \\
 Q \langle c, c \rangle
   \arrow[dr, "{P \langle c, f \rangle}"']
 &&  Q \langle c', c' \rangle
 \arrow[dl,"{P \langle f, c \rangle}"]
\\
&Q \langle c, c' \rangle
 \end{tikzcd}
\]
(I used the common shorthand $P \langle f, c \rangle$ for $P \langle f, id_c \rangle$.)

In our case, the dinaturality condition simplifies to:
\[
 \begin{tikzcd}
 & P \langle a, b \rangle
 \arrow[dl, "{\alpha_{\langle a, b \rangle, c}}"']
 \arrow[dr, "{\alpha_{\langle a, b \rangle, c'}}"]
 \\
 P \langle c \times a, c \times b \rangle
   \arrow[dr, "{P \langle c \times a, f \times b \rangle}"']
 &&  P \langle c' \times a, c'  \times b\rangle
 \arrow[dl,"{P \langle f \times b, c \times b \rangle}"]
\\
&P \langle c \times a, c' \times b \rangle
 \end{tikzcd}
\]
(Here, again $P \langle f \times b, c \times b \rangle$ stands for $P \langle f \times id_b, id_{c \times b} \rangle$.)

There is one more consistency condition on Tambara modules: they must preserve the monoidal structure. The action of multiplying by $c$ makes sense in a cartesian category: we have to have a product for any pair of objects, and we want to have the terminal object to serve as the unit of multiplication. Tambara modules have to respect unit and preserve multiplication. For the unit (terminal object), we impose the following condition:
\[ \alpha_{\langle a, b \rangle, 1} = id _{P \langle a, b \rangle}\]
For multiplication, we have:
\[ \alpha_{\langle a, b \rangle, c' \times c} \cong  \alpha_{\langle c \times a, c \times b \rangle, c'} \circ  \alpha_{\langle a, b \rangle, c}\]
or, pictorially:
\[
 \begin{tikzcd}
 P \langle a, b \rangle
 \arrow[rr, "{\alpha_{\langle a, b \rangle, c' \times c } }"]
 \arrow[rdd, "{ \alpha_{\langle a, b \rangle, c}}"']
 &&
 P \langle c' \times c \times a, c' \times c \times b \rangle
 \\
 \\
 & P \langle c \times a, c \times b \rangle
  \arrow[ruu, "{\alpha_{\langle c \times a, c \times b \rangle, c'}}"']
\end{tikzcd}
\]
Notice that the product is associative up to isomorphism, so there is a hidden associator in this diagram.

Since we want Tambara modules to form a category, we have to define morphisms between them. These are natural transformations that preserve the additional structure. Let's say we have a natural transformation between two Tambara modules $\rho \colon (P, \alpha) \to (Q, \beta) $. We can either apply $\alpha$ and then $\rho$, or do $\rho$ first and then  $\beta$. We want the result to be the same:
\[
 \begin{tikzcd}
  P \langle a, b \rangle
 \arrow[d, "{ \rho_{\langle a, b \rangle}}"']
 \arrow[r, "{ \alpha_{\langle a, b \rangle, c}}"]
  &  P \langle c \times a, c \times b \rangle
  \arrow[d, "{ \rho_{\langle c \times a, c \times b \rangle}}"]
\\
   Q \langle a, b \rangle
 \arrow[r, "{ \beta_{\langle a, b \rangle, c}}"]
 &  Q \langle c \times a, c \times b \rangle
 \end{tikzcd}
\]


\subsection{Profunctor lenses}

Now that we have some intuition about how Tambara modules are related to lenses, let's go back to our main formula:
\[  \mathcal{L}\langle s, t\rangle \langle a, b \rangle =\int_{P \colon \cat T} \Set \big((U P) \langle a, b\rangle, (U P) \langle s, t\rangle \big) \cong \big( \Phi (\cat C^{op} \times \cat C) (\langle a, b\rangle ,-) \big) \langle s, t\rangle \]
This time we're taking the end over the Tambara category. The only missing part is the monad $\Phi = U \circ F$ or the functor $F$ that freely generates Tambara modules.

It turns out that, instead of guessing the monad, it's easier to guess the comonad. There is a comonad in the category of profunctors that takes a profunctor $P$ and produces another profunctor $\Theta P$. Here's the formula:
\[(\Theta P) \langle a, b \rangle = \int_c P \langle c \times a, c \times b \rangle \]
You can check that this is indeed a comonad by implementing $\varepsilon$ and $\delta$ (\hask{extract} and \hask{duplicate}). For instance, $\varepsilon$ maps $\Theta P \to P$ using the projection $\pi_1$ for the terminal object (the unit of cartesian product).

What's interesting about this comonad is that its coalgebras are Tambara modules. Again, these are coalgebras that map profunctors to profunctors. They are natural transformations $P \to \Theta P$. We can write such a natural transformation as an end:
\[\int_{\langle a, b \rangle} \Set \big (P \langle a, b \rangle, (\Theta P) \langle a, b \rangle \big) = \int_{\langle a, b \rangle} \int_c \Set \big( P\langle a, b \rangle , P \langle c \times a, c \times b \rangle \big) \]
I used the continuity of the hom-functor to pull out the end over $c$. The resulting end encodes a set of natural (dinatural in $c$) transformations that define a Tambara module:
\[ \alpha_{\langle a, b\rangle, c} \colon P \langle a, b \rangle \to P \langle c \times a, c \times b \rangle \]
In fact, these coalgebras are comonad coalgebras, that is they are compatible with the comonad $\Theta$. In other words, Tambara modules form the Eilenberg-Moore category of coalgebras for the comonad $\Theta$.

The left adjoint to $\Theta$ is a monad $\Phi$ given by the formula:
\[(\Phi P)  \langle s, t \rangle = \int^{\langle u, v \rangle, c} (\cat C^{op} \times \cat C) \big(c \bullet \langle u, v\rangle , \langle s, t \rangle\big) \times P \langle u, v \rangle \]
where:
\[ (\cat C^{op} \times \cat C) \big(c \bullet \langle u, v\rangle , \langle s, t \rangle\big) = \cat C(s, c \times s) \times \cat C(c \times v, t) \]

This adjunction can be easily checked using some end/coend manipulations. The mapping out of $\Phi P$ to some profunctor $Q$ can be written as an end. The coends in $\Phi$ can then be taken out using co-continuity of the hom-functor. Finally, applying the ninja-Yoneda lemma produces the mapping into $\Theta Q$. We get:
\[ [(\cat C^{op} \times \cat C, \Set] (P \Phi, Q) \cong [(\cat C^{op} \times \cat C, \Set] (P, \Theta Q) \]

Replacing $Q$ with $P$ we immediately see that the set of algebras for $\Phi$ is isomorphic to the set of coalgebras for $\Theta$. In fact they are monad algebras for $\Phi$. This means that the Eilenberg-Moore category for the monad $\Phi$ is the same as the Tambara category.

Recall that the Eilenberg-Moore construction factorizes a monad into a free/forgetful adjunction. This is exactly the adjunction we were looking for in deriving the formula for profunctor optics. What remains is to evaluate the action of $\Phi$ on the representable functor:
\[ \big( \Phi (\cat C^{op} \times \cat C) (\langle a, b\rangle ,-) \big) \langle s, t\rangle = \int^{\langle u, v \rangle, c} (\cat C^{op} \times \cat C) \big(c \bullet \langle u, v\rangle , \langle s, t \rangle \big) \times  (\cat C^{op} \times \cat C) \big(\langle a, b\rangle , \langle u, v\rangle \big)\]
Applying the co-Yoneda lemma, we get:
\[ \int^c (\cat C^{op} \times \cat C) \big(c \bullet \langle a, b\rangle , \langle s, t \rangle\big) = \int^c \cat C(s, c \times a) \times \cat C (c \times b, t)\]
which is exactly the existential representation of the lens.

\subsection{Profunctor lenses in Haskell}

To define profunctor representation in Haskell we introduce a class of profunctors that are Tambara modules with respect to cartesian product (we'll see more general Tambara modules later). In the Haskell library this class is called \hask{Strong}. It also appears in the literature as \hask{Cartesian}:
\begin{haskell}
class Profunctor p => Cartesian p where
  alpha :: p a b -> p (c, a) (c, b)
\end{haskell}
The polymorphic function \hask{alpha} has all the relevant naturality properties guaranteed by parametric polymorphism. 

The profunctor lens is just a type synonym for a function type that is polymorphic in \hask{Cartesian} profunctors:
\begin{haskell}
type LensP s t a b = forall p. Cartesian p => p a b -> p s t
\end{haskell}

The easiest way to implement such a function is to start from the existential representation of a lens and apply \hask{alpha} followed by \hask{dimap} to the profunctor argument:
\begin{haskell}
toLensP :: Lens s t a b -> LensP s t a b
toLensP (Lens from to) = dimap from to . alpha
\end{haskell}

Because profunctor lenses are just functions, we can compose them as such:
\begin{haskell}
lens1 :: LensP s t x y 
-- p s t -> p x y
lens2 :: LensP x y a b 
-- p x y -> p a b
lens3 :: LensP s t a b 
-- p s t -> p a b
lens3 = lens2 . lens1
\end{haskell}

\section{General Optics}
Tambara modules were originally defined for an arbitrary monoidal category\footnote{In fact, Tambara modules were originally defined for an enriched category} with a tensor product $\otimes$ and a unit object $I$. Their structure maps have the form:
\[ \alpha_{\langle a, b\rangle, c} \colon P \langle a, b \rangle \to P \langle c \otimes a, c \otimes b \rangle \]
You can easily convince yourself that all coherency laws translate directly to this case, and the derivation of profunctor optics works without change.

\subsection{Prisms}

From the programming point of view there are two obvious monoidal structures to explore: the product and the sum. We've seen that the product gives rise to lenses. The sum, or the coproduct, gives rise to prisms. 

We get the existential representation simply by replacing the product by the sum in the definition of a lens:
\[ \mathcal{P}\langle s, t\rangle \langle a, b \rangle = \int^{c} \mathcal{C}(s, c + a) \times  \mathcal{C}(c + b, t) \]
To simplify this, notice that the mapping out of a sum is equivalent to the product of mappings:
\[ \int^{c} \mathcal{C}(s, c + a) \times  \mathcal{C}(c + b, t) \cong  \int^{c} \mathcal{C}(s, c + a) \times  \mathcal{C}(c, t) \times  \mathcal{C}(b, t) \]
Using the co-Yoneda lemma, we can get rid of the coend to get:
\[ \mathcal{C}(s, t + a) \times  \mathcal{C}(b, t) \]

In Haskell, this defines a pair of functions:
\begin{haskell}
match :: s -> Either t a
build :: b -> t
\end{haskell}

To understand this, let's first translate the existential form of the prism:
\begin{haskell}
data Prism s t a b where
  Prism :: (s -> Either c a) -> (Either c b -> t) -> Prism s t a b
\end{haskell}
Here \hask{s} either contains the focus \hask{a} or the residue \hask{c}. Conversely, \hask{t} can be built either from the new focus \hask{b}, or from the residue \hask{c}. 

This logic is reflected in the conversion functions:
\begin{haskell}
toMatch :: Prism s t a b -> (s -> Either t a)
toMatch (Prism from to) s =
  case from s of
    Left  c -> Left (to (Left c))
    Right a -> Right a
\end{haskell}

\begin{haskell}
toBuild :: Prism s t a b -> (b -> t)
toBuild (Prism from to) b = to (Right b)
\end{haskell}

\begin{haskell}
toPrism :: (s -> Either t a) -> (b -> t) -> Prism s t a b
toPrism match build = Prism from to
  where
    from = match
    to (Left  c) = c
    to (Right b) = build b
\end{haskell}

The profunctor representation of the prism is almost identical to that of the lens, except for swapping the product for the sum. 

The class of Tambara modules for the sum type is called \hask{Choice} in the Haskell library, or \hask{Cocartesian} in the literature:
\begin{haskell}
class Profunctor p => Cocartesian p where
  alpha' :: p a b -> p (Either c a) (Either c b)
\end{haskell}
The profunctor representation is a polymorphic function type:
\begin{haskell}
type PrismP s t a b = forall p. Cocartesian p => p a b -> p s t
\end{haskell}

The conversion from the existential prism is virtually identical to that of the lens:
\begin{haskell}
toPrismP :: Prism s t a b -> PrismP s t a b
toPrismP (Prism from to) = dimap from to . alpha'
\end{haskell}

Again, profunctor prisms compose using function composition.
\subsection{Traversals}

\begin{haskell}
\end{haskell}

\end{document}
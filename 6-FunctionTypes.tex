\documentclass[DaoFP]{subfiles}
\begin{document}
\setcounter{chapter}{5}

\chapter{Function Types}

There is one kind of composition that is at heart of functional programming. It happens when you pass a function as an argument to another function. If we model functions as arrows between objects, what does it mean to have a function as an argument? We need a way to objectify functions. We need an ``object of arrows'' that can serve as a source for another arrow. We also need arrows whose target is the object of arrows. These will model functions that return other functions.

The defining quality of a function is that it can be applied to an argument to produce the result. We defined function application in terms of composition:

\[
 \begin{tikzcd}
 1
 \arrow[d, "a"']
 \arrow[rd, "b"]
 \\
 A
 \arrow[r, "f"']
& B
 \end{tikzcd}
\]
Here $f$ is represented as an arrow from $A$ to $B$, but we would like to work with $f$ as an element of the object of arrows, or the exponential $B^A$. Given an element $f$ of $B^A$ and an element $a$ of $A$, the application should produce an element $b$ of $B$. A pair of elements is equivalent to an element of a product, so we'd like to define the application as an arrow:
\[\text{app} \colon B^A \times A \to B\]
This way, well have:
\[
 \begin{tikzcd}
 1
 \arrow[d, "{(f, a)}"']
 \arrow[rd, "b"]
 \\
 B^A \times A
 \arrow[r, "\text{app}"']
& B
 \end{tikzcd}
\]



\section{notes}



\begin{exercise}
\end{exercise}
\begin{haskell}
\end{haskell}



\end{document}
\documentclass[11pt, book]{memoir}

\settrims{0pt}{0pt} % page and stock same size
\settypeblocksize{*}{34.5pc}{*} % {height}{width}{ratio}
\setlrmargins{*}{*}{1} % {spine}{edge}{ratio}
\setulmarginsandblock{1in}{1in}{*} % height of typeblock computed
\setheadfoot{\onelineskip}{2\onelineskip} % {headheight}{footskip}
\setheaderspaces{*}{1.5\onelineskip}{*} % {headdrop}{headsep}{ratio}
\checkandfixthelayout

\chapterstyle{bianchi}
\newcommand{\titlefont}{\normalfont\Huge\bfseries}
\renewcommand{\chaptitlefont}{\titlefont}

\usepackage{subfiles}
\usepackage{makeidx}

\usepackage{amsfonts}
\usepackage{amssymb}  
\usepackage{amsthm} 
\usepackage{amsmath} 
\usepackage{tikz-cd}
\usetikzlibrary{backgrounds}
\usepackage{float}
\usepackage{xcolor}

\newtheorem{exercise}{Exercise}[section]
\newcommand{\exinline}[1]{(\refstepcounter{exercise}Exercise~\theexercise\label{#1})}

\usepackage{minted}
\usepackage{etoolbox}

\usepackage{hyperref}
\hypersetup{
  colorlinks,
  linkcolor={red!40!black},
  citecolor={red!40!black},
  urlcolor={blue!40!black}
}

\newcommand{\cat}[1]{\mathcal{#1}}%a generic category
\newcommand{\Cat}[1]{\mathbf{#1}}%a named category
\newcommand{\Set}{\Cat{Set}}

\newcommand{\hask}[1]{\mintinline{Haskell}{#1}}
\newenvironment{haskell}
  {\VerbatimEnvironment
  	%\begin{minted}[escapeinside=??, mathescape=true, frame=single, framesep=5pt, tabsize=1, bgcolor=black!5]{Haskell}}
  	\begin{minted}[escapeinside=??, mathescape=true, tabsize=1, bgcolor=black!3, xleftmargin=3pt ]{Haskell}}
  {\end{minted}}
\makeatletter
% replace \medskip before and after the box with nothing, i.e., remove it
\patchcmd{\minted@colorbg}{\medskip}{}{}{}
\patchcmd{\endminted@colorbg}{\medskip}{}{}{}
\makeatother

\makeindex

\begin{document}
\setcounter{tocdepth}{4}
\setcounter{secnumdepth}{4}
\frontmatter

\title{\huge The Dao of Functional Programming}
\author{\Large Bartosz Milewski }

\date{\vfill (Last updated: \today)}

\maketitle

\tableofcontents*

\clearpage

\section{Preface}

Most programming texts, following Brian Kernighan, start with ``Hello World!''. It's natural to want to get the immediate gratification of making the computer do your bidding and print these famous words. But the real mastery of computer programming goes deeper than that, and rushing into it may only give you a false feeling of power, when in reality you're just parroting the masters. If your ambition is just to learn a useful, well-paid skill then, by all means, write your "Hello World!" program. There are tons of books and courses that will teach you how to write code in any language of your choice. However, if you really want to get to the essence of programming, you need to be patient and persistent.

Category theory is the branch of mathematics that provides the abstractions that accord with the practical experience of programming. Paraphrasing von Clausewitz: Programming is merely the continuation of mathematics with other means. A lot of complex ideas of category theory become obvious to programmers when explained in terms of data types and functions. In this sense, category theory might be more accessible to programmers than it is to professional mathematicians.

When faced with a new categorical concepts I would often look them up on Wikipedia or nLab, or re-read a chapter from Mac Lane or Kelly. These are great sources, but they require some up front familiarity with the topics and the ability to fill in the gaps. One of the goals of this book is to provide the necessary bootstrap to continue studying category theory.

There is a lot of folklore knowledge in category theory and in computer science that is nowhere to be found in the literature. It's very difficult to acquire useful intuitions when going through dry definitions and theorems. I tried, as much as possible, to provide the missing intuitions and explain not only the whats but also the whys.

The title of this book alludes to Benjamin Hoff's ``The Tao of Pooh'' and to Robert Pirsig's ``Zen and the Art of Motorcycle Maintenance,'' both being attempts by Westerners to assimilate elements of Easter philosophy. Loosely speaking, the idea is that category theory is to programming as the Dao\footnote{Dao is the more modern spelling of Tao} is to Western philosophy. Many of the definitions of category theory make no sense on first reading but in time you learn to appreciate their deep wisdom. If category theory were to be summarized in one soundbite, it would be: ``Things are defined by they relationship to the Universe.'' 

\subsection{Set theory}

Some familiarity with the basics of set theory is necessary to learn category theory. For instance the idea that sets contain elements. We say that, given a set $S$ and an element $a$, it makes sense to ask whether $a$ is an element of $S$ or not. This is written as $a \in S$. It's also possible to have an empty set that contains no elements. 

The important property of elements of a set is that they can be compared for equality. Given two elements $a \in S$ and $b \in S$, we can ask: Is $a$ equal to $b$? Or we may impose the condition that $a = b$, in case $a$ and $b$ are results of two different recipes for selecting elements of the set $S$.

A cartesian product of two sets $S \times T$ is defined as a set of all pairs of elements $\langle s, t\rangle$, such that $s \in S$ and $t \in T$.

A function $f \colon S \to T$ from the source set called the \emph{domain} of $f$ to the target set called the \emph{codomain} is also defined as a set of pairs. These are the pairs of the form $\langle s, t \rangle$ where $t = f s$. Here $f s$ is the result of the action of $f$ on the argument $s$. You might be more familiar with the notation $f(s)$ for function application, but here I'll follow the Haskell convention of omitting the parentheses (and commas, for functions of multiple variables).

In programming we are used to functions being defined by a sequence of instructions. We provide an argument $s$ and apply the instructions to eventually produce the result $t$. We are often worried about how long it my take to evaluate the result, or if the algorithm terminates at all. In mathematics we assume that, for any given argument $s \in S$ the result $t \in T$ is immediately available, and that it's unique. In programming we call such functions pure and total.

\subsection{Conventions}

I tried to keep the notation coherent throughout the book. It's based loosely on the prevailing style in nLab.

In particular, I decided to use lowercase letters like $a$ or $b$ for objects in a category and uppercase letters like $S$ for sets (even though sets are objects in the category of sets and functions). Generic categories have names like $\cat C$ or $\cat D$, whereas specific categories have names like $\Set$ or $\Cat{Cat}$. 

Programming examples are written in Haskell. Although this is not a Haskell manual, the introduction of language constructs is gradual enough to help the reader navigate the code. The fact that Haskell syntax is often based on mathematical notation is an additional advantage. Program fragments are written in the following format:
\begin{haskell}
apply :: (a -> b, a) -> b
apply (f, x) = f x
\end{haskell}

\mainmatter

\subfile{1-CleanSlate}
\subfile{2-Composition}
\subfile{3-Isomorphism}
\subfile{4-SumTypes}
\subfile{5-ProductTypes}
\subfile{6-FunctionTypes}
\subfile{7-Recursion}
\subfile{8-Functors}
\subfile{9-NaturalTransformations}
\subfile{10-Adjunctions}
\subfile{11-DependentTypes}
\subfile{12-Algebras}
\subfile{13-Coalgebras}
\subfile{14-Monads}
\subfile{15-MonadsAdjunctions}
\subfile{16-Comonads}
\subfile{17-Ends}
\subfile{18-Tambara}
\subfile{19-Kan}
\subfile{20-Enrichment}

\printindex

\end{document}
\documentclass[11pt, book]{memoir}

\settrims{0pt}{0pt} % page and stock same size
\settypeblocksize{*}{34.5pc}{*} % {height}{width}{ratio}
\setlrmargins{*}{*}{1} % {spine}{edge}{ratio}
\setulmarginsandblock{1in}{1in}{*} % height of typeblock computed
\setheadfoot{\onelineskip}{2\onelineskip} % {headheight}{footskip}
\setheaderspaces{*}{1.5\onelineskip}{*} % {headdrop}{headsep}{ratio}
\checkandfixthelayout

\chapterstyle{bianchi}
\newcommand{\titlefont}{\normalfont\Huge\bfseries}
\renewcommand{\chaptitlefont}{\titlefont}

\usepackage{subfiles}
\usepackage{makeidx}

\usepackage{amsfonts}
\usepackage{amssymb}  
\usepackage{amsthm} 
\usepackage{amsmath} 
\usepackage{tikz-cd}
\usetikzlibrary{backgrounds}
\usepackage{float}
\usepackage{xcolor}

\newtheorem{exercise}{Exercise}[section]
\newcommand{\exinline}[1]{(\refstepcounter{exercise}Exercise~\theexercise\label{#1})}


\usepackage{minted}
\usepackage{etoolbox}

\newcommand{\cat}[1]{\mathcal{#1}}%a generic category
\newcommand{\Cat}[1]{\mathbf{#1}}%a named category
\newcommand{\Set}{\Cat{Set}}

\newcommand{\hask}[1]{\mintinline{Haskell}{#1}}
\newenvironment{haskell}
  {\VerbatimEnvironment
  	%\begin{minted}[escapeinside=??, mathescape=true, frame=single, framesep=5pt, tabsize=1, bgcolor=black!5]{Haskell}}
  	\begin{minted}[escapeinside=??, mathescape=true, tabsize=1, bgcolor=black!3, xleftmargin=3pt ]{Haskell}}
  {\end{minted}}
\makeatletter
% replace \medskip before and after the box with nothing, i.e., remove it
\patchcmd{\minted@colorbg}{\medskip}{}{}{}
\patchcmd{\endminted@colorbg}{\medskip}{}{}{}
\makeatother

\makeindex

\begin{document}
\setcounter{tocdepth}{4}
\setcounter{secnumdepth}{4}
\frontmatter

\title{\huge The Dao of Functional Programming}
\author{\Large Bartosz Milewski }

\date{\vfill (Last updated: \today)}

\maketitle

\tableofcontents*

\clearpage

\section{Preface}

Most programming texts, following Brian Kernighan, start with "Hello World!". It's natural to want to get the immediate gratification of making the computer do your bidding and print these famous words. But the real mastery of computer programming goes deeper than that, and rushing into it may only give you a false feeling of power, when in reality you're just parroting the masters. If your ambition is just to learn a useful, well-paid skill then, by all means, write your "Hello World!" program. There are tons of books and courses that will teach you to write code in any language of your choice. However, if you really want to get to the essence of programming, you need to be patient and persistent.

\subsection{Conventions}

I tried to keep the notation coherent throughout the book. In particular, I decided to use lowercase letters like $a$ or $b$ for objects in a category and uppercase names like $S$ for sets, even though sets are objects in the category of sets and functions. Generic categories have names like $\cat C$ or $\cat D$, whereas specific categories are named like $\Set$ or $\Cat{Cat}$. 

Programming examples are written in Haskell. Although this is not a Haskell manual, the introduction of language constructs is gradual enough to help the reader navigate the code. The fact that Haskell syntax is often based on mathematical notation is an additional advantage. Program fragments are written in the following format:
\begin{haskell}
apply :: (a -> b, a) -> b
apply (f, x) = f x
\end{haskell}

\mainmatter

\subfile{1-CleanSlate}
\subfile{2-Composition}
\subfile{3-Isomorphism}
\subfile{4-SumTypes}
\subfile{5-ProductTypes}
\subfile{6-FunctionTypes}
\subfile{7-Recursion}
\subfile{8-Functors}
\subfile{9-NaturalTransformations}
\subfile{10-Adjunctions}
\subfile{11-DependentTypes}
\subfile{12-Algebras}
\subfile{13-Coalgebras}
\subfile{14-Monads}
\subfile{15-MonadsAdjunctions}
\subfile{16-Comonads}
\subfile{17-Ends}
\subfile{18-Tambara}
\subfile{19-Kan}
\subfile{20-Enrichment}

\printindex

\end{document}